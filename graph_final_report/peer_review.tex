\section{ {Mixed Graph Signal Analysis of Joint Image Denoising / Interpolation \cite{10445943}} についての査読レポート}

\subsection{Strengths}

This paper presents three noteworthy contributions to the field of image processing: a novel theoretical framework connecting linear operations to graph filters, a method for joint optimization of denoising and interpolation, and empirical validation of performance gains. Each will be discussed in turn.

First, the theoretical framework established in this paper is particularly significant. The authors present two key theorems that bridge the gap between traditional linear image processing operations and graph-based approaches. Theorem 1 demonstrates that under certain conditions, any linear denoiser can be interpreted as the solution to a maximum a posteriori optimization problem using an undirected graph Laplacian regularizer. This result provides a new lens through which to analyze and potentially improve denoising operations using graph-theoretic techniques. Similarly, Theorem 2 extends this concept to linear interpolators, showing that they can be formulated as solutions to optimization problems on directed graphs. These theorems are remarkable in their ability to unify seemingly disparate approaches to image processing, offering a common mathematical language for both classical and graph-based methods.

Second, building on this theoretical foundation, the authors develop a method for joint optimization of denoising and interpolation operations. They derive analytical expressions for optimal filters in both separable and non-separable cases. This contribution is particularly valuable as it provides practical guidance on when to apply joint optimization versus sequential processing. The ability to determine the optimal approach for a given scenario can lead to more efficient image processing pipelines. Moreover, the authors' derivation of closed-form expressions for these optimal joint filters is a significant achievement, offering not only theoretical insight but also practical tools for implementation.

Third, the paper provides robust empirical validation of the proposed framework through a series of experiments. The authors test their method on standard image processing tasks such as rotation and warping under various noise levels. The results, clearly illustrated in Figure 2, demonstrate consistent performance improvements over sequential application of denoising and interpolation. With PSNR gains of up to 1.35 dB in some cases, these experimental findings provide strong evidence for the practical utility of the proposed approach. This empirical validation strengthens the paper's contributions by showing that the theoretical insights translate into tangible improvements in real-world image processing tasks.

Furthermore, the paper's potential impact on future research directions is significant. By establishing a rigorous connection between linear operations and graph filters, the authors have opened up new avenues for analyzing and improving image processing techniques. This bridge between traditional and graph-based approaches could potentially lead to the development of hybrid methods that combine the strengths of both paradigms. Such hybrid approaches might offer improved performance in challenging scenarios where neither traditional nor pure graph-based methods suffice.

Additionally, the framework presented in this paper could potentially be extended to more complex image processing tasks. While the current work focuses on denoising and interpolation, the underlying principles might be applicable to other operations such as image segmentation, feature extraction, or even high-level computer vision tasks. This extensibility makes the paper's contributions particularly valuable, as they provide a foundation for future innovations in various areas of image processing and analysis.

\subsection{Weaknesses}

While the paper presents significant theoretical contributions, there are some limitations and areas that could benefit from further development. One limitation is the focus on linear denoisers and interpolators. Many state-of-the-art image processing methods use non-linear or learning-based approaches. It's unclear how well the proposed framework would extend to these more complex techniques. The authors could strengthen the work by discussing potential extensions to non-linear cases or providing some preliminary results in this direction.

The experimental validation, while demonstrating clear improvements, is somewhat limited in scope. The authors test their method on only two standard test images (Lena and Peppers) and a small set of operations (rotation and warping). A more comprehensive evaluation on a larger dataset of diverse images and a wider range of transformations would provide stronger evidence for the generality of the proposed approach. Additionally, comparisons against more recent state-of-the-art methods for joint denoising and interpolation would help contextualize the performance gains.

Another potential weakness is the lack of theoretical analysis of the computational complexity of the proposed joint optimization approach. While the authors mention using conjugate gradient to solve the linear system, they don't provide runtime comparisons or discuss the scalability of their method to larger images or 3D volumes. This information would be valuable for assessing the practical applicability of the technique.

The paper also leaves some open questions regarding parameter selection. The authors state that "For the experiments with Gaussian filter and BF, the hyperparameters were selected as μ = 0.3, γ = 0.5, κ = 0.3." However, there's no discussion of how sensitive the results are to these choices or guidance on how to select optimal parameters for new scenarios. A more detailed analysis of parameter sensitivity would strengthen the practical utility of the method.

Furthermore, while the theoretical results are mathematically rigorous, the paper could benefit from more intuitive explanations or visual illustrations of why the graph-based formulations lead to improved performance. Additional discussion of the practical implications and insights gained from interpreting denoisers and interpolators as graph filters would enhance the paper's impact.

Lastly, the authors mention potential applications in areas like demosaicing and image rectification in the introduction, but don't return to these topics in the results or conclusion. Exploring how the proposed framework could be applied to these more complex real-world scenarios would strengthen the motivation and broaden the appeal of the work.

In summary, while the paper makes valuable theoretical contributions and demonstrates promising empirical results, there is room for expansion in terms of experimental validation, analysis of computational aspects, and exploration of practical applications. Addressing these points could further enhance the impact and applicability of the proposed approach.

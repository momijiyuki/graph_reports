\section{グラフ信号処理/グラフニューラルネットワークを自身の研究分野で利用するとしたら,どのような応用分野が考えられるか?}

現在の研究分野は決定木学習における、決定木の最適化である。決定木は閉路を持たない有向なグラフ構造といえる。
そのため、ノードの数の辺を持つ隣接行列を考えることができる。決定木は、各ノードにおいてデータ集合を2つの部分集合に分割していく。この時のデータ集合の数を各ノードの結合の重みとして表現することもできると考える。そうしてグラフフーリエ変換を適用することで、高周波成分としてテストデータに対して部分的に反応するノードが見つかると考える。
\footnote{
    通常、決定木はノードが深くなるにつれ分類時に参照される確率は低くなっていくため、ノードの深さ $n$ による $\frac{1}{2^n}$ や、該当ノードに含まれる学習データ数で割ったりして正規化することで正しく比較できると考える
}
局所的に反応するノードは過適合を起こす可能性のあるノードとして枝刈りや部分的な再学習をすることで全体的な精度やロバスト性の向上が期待できるかもしれない。または、通常決定木は貪欲的に情報利得を最大化する位置で分割を行っているが、局所的に反応するノードにあまりデータを送らないよう範囲を狭めたり、逆にデータを多く送るようにして再学習することでノードでの偏りを抑え、精度向上に繋がる可能性もあると考える。また、注釈にあるように正規化した頻度情報の逆数を重みとしてBoostingの手法を適用することも可能かもしれない。

ここまで閉路のないグラフとして考えていたが、分類問題において、決定木の葉ノードにはクラスの重複が発生する。この同じクラスと識別した葉ノード同士を(強い・弱い)結合の無向グラフとして考えれば閉路が生まれ、通常のグラフに近い構造となる。そうしてグラフフーリエ変換等を適用することでなにかしらの知見が得られるのではないかと考える。
